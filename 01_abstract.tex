\documentclass[a4paper,11pt,uplatex]{jsbook}

%\usepackage{fancyhdr}
\setlength{\footskip}{16pt}
\usepackage{amsmath}
\usepackage[dvipdfmx]{graphicx}
\usepackage[dvipdfmx]{color}
%\usepackage{pagecolor}[white]
\usepackage{amsmath,amssymb}
%\usepackage[top=3cm, bottom=3cm, left=3cm, right=3cm]{geometry}
\usepackage{braket}
\usepackage{bm}
\numberwithin{equation}{section}
\usepackage{mathrsfs}
\usepackage{siunitx}
\usepackage{physics}
\usepackage[dvipdfmx]{graphicx}
\usepackage[compat=1.1.0]{tikz-feynhand}
\usepackage{caption}
\usepackage{subcaption}
%\usepackage{cleveref}
\usepackage{float}
\usepackage{multicol}
\setlength{\columnsep}{15mm}
%\usepackage[style=phys,articletitle=false,biblabel=brackets,chaptertitle=false,pageranges=false]{biblatex}
%\usepackage[style=phys]{biblatex}
\usepackage[dvipdfmx]{hyperref}
\usepackage{url}
\usepackage{pxjahyper}
\usepackage{bookmark}
%\usepackage[backref]{hyperref}
\setcounter{tocdepth}{3}
\setlength{\parindent}{2em}
\def\vector#1{\mbox{\boldmath $#1$}}
\def\slash#1{\not\!#1}
\def\slashb#1{\not\!\!#1}
\def\delsla{\not\!\partial}
\usepackage[dvipdfmx]{xcolor}


\hypersetup{
 setpagesize=false,
 bookmarksnumbered=true,%
 bookmarksopen=true,%
 colorlinks=true,%
 linkcolor=black,
 citecolor=red,
 urlcolor=black,
}
%backreferenceのカスタマイズ. "Back to p.3"のように表示する.
%\renewcommand*{\backref}[1]{(p.#1へ戻る)}
%\newcommand{\backtoc}{\hyperlink{toc}{[目次へ]}}
\newcommand{\backtoc}{\texorpdfstring{\protect\hyperlink{toc}{\hspace{5pt} \scriptsize [目次へ]}}{}}
\newcommand{\mychapter}[1]{\chapter[#1]{#1\backtoc}}
\newcommand{\mysection}[1]{\section[#1]{#1\backtoc}}
\newcommand{\mysubsection}[1]{\subsection[#1]{#1\backtoc}}
% 数式
%\usepackage{amsmath,amsfonts}
%\usepackage{bm}
%\usepackage{physics}
%\usepackage{siunitx}
% 画像
%\usepackage[dvipdfmx]{graphicx}
%\usepackage[dvipdfmx,colorlinks=true,linkcolor=blue]{hyperref}
%\usepackage{pxjahyper}

\begin{document}
\chapter*{概要}
通常原子核は陽子と中性子からなるが、陽子、中性子はさらにu,dクォークという素粒子からなる。u,dに次いで重いsクォークを含んだバリオンを
ハイペロンと呼び、核子だけでなくハイペロンを含む原子核をハイパー核と呼ぶ。
ハイパー核を実験的に生成し、その崩壊過程で放出される粒子を測定することで、ハイペロンと核子の間に働く強い相互作用、核力を調べることができる。
ハイペロンを含めた核子間相互作用の研究は、バリオン間力の統一的な理解に極めて重要である。

もっとも基本的なハイペロンはu,d,sクォークを一つずつ含んだ$\Lambda$粒子であり、1960年代から$\Lambda$粒子を含むハイパー核の研究が進められてきた。
近年、A=3体系のもっとも基本的なハイパー核であるハイパートライトン$_{\Lambda}^3\text{H}$において、$\Lambda$粒子の束縛がこれまでの理解では
説明できない程度に弱いことを示唆する実験結果が報告された。
ハイパートライトンをはじめとして軽い$\Lambda$ハイパー核における$\Lambda$粒子の束縛エネルギーを精密に測定することがますます重要になっている。

こうした軽い$\Lambda$ハイパー核の精密質量分光の手法として、我々の研究グループではドイツ・マインツ大学において崩壊パイ中間子法と呼ばれる手法を開発した。
2012年および2014年に行った実験では、$_{\Lambda}^4\text{H}$の基底状態における$\Lambda$粒子の束縛エネルギーを10~kev以下の統計誤差で決定した。
一方で、磁気スペクトロメータの較正に起因する系統誤差が70~keVあり、課題とされていた。
この実験では電子弾性散乱実験を行うことで磁気スペクトロメータの較正を行っているが、この精度は主に電子ビームエネルギーの決定精度に起因している。
$\Lambda$粒子の束縛エネルギーの系統誤差を統計誤差と同水準に抑えるためには、電子弾性散乱に用いる200~MeV領域の電子ビームエネルギーの決定精度を
従来の10倍以上向上させ、$20$~keV以下の精度で測定する必要がある。

200~MeV領域の電子ビームエネルギーを$20$~keV以下の精度で測定する手法は限定的であり、
我々の研究グループでは新たにアンジュレータ放射光干渉法を開発した。
この手法では、アンジュレータと呼ばれる放射光源を2つ用いて、2つのアンジュレータから放射される放射光の干渉光強度を測定することで電子ビームエネルギーを決定する。
2024年3月-5月にかけて、アンジュレータ放射光干渉法を併用した電子弾性散乱によるスペクトロメータ較正実験を行った。
本論文は、この実験における電子ビームエネルギー測定の概要と結果についてまとめたものである。

放射光の振幅・位相の空間分布を計算したうえで、光学系内の光波の伝搬を計算するモデル関数を構成し、観測された干渉光の画像の再現を行った。
この過程では、アンジュレータと光学系の位置関係を反映させたパラメータの最適化を行った。
最終的に、今回目標とした測定精度を満たす電子ビームエネルギーの測定精度を得ることができた。

\end{document}