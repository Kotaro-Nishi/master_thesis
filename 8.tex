\documentclass[a4paper,11pt,uplatex]{jsbook}
%\usepackage{fancyhdr}
\setlength{\footskip}{16pt}
\usepackage{amsmath}
\usepackage[dvipdfmx]{graphicx}
\usepackage[dvipdfmx]{color}
%\usepackage{pagecolor}[white]
\usepackage{amsmath,amssymb}
%\usepackage[top=3cm, bottom=3cm, left=3cm, right=3cm]{geometry}
\usepackage{braket}
\usepackage{bm}
\numberwithin{equation}{section}
\usepackage{mathrsfs}
\usepackage{siunitx}
\usepackage{physics}
\usepackage[dvipdfmx]{graphicx}
\usepackage[compat=1.1.0]{tikz-feynhand}
\usepackage{caption}
\usepackage{subcaption}
%\usepackage{cleveref}
\usepackage{float}
\usepackage{multicol}
\setlength{\columnsep}{15mm}
%\usepackage[style=phys,articletitle=false,biblabel=brackets,chaptertitle=false,pageranges=false]{biblatex}
%\usepackage[style=phys]{biblatex}
\usepackage[dvipdfmx]{hyperref}
\usepackage{url}
\usepackage{pxjahyper}
\usepackage{bookmark}
%\usepackage[backref]{hyperref}
\setcounter{tocdepth}{3}
\setlength{\parindent}{2em}
\def\vector#1{\mbox{\boldmath $#1$}}
\def\slash#1{\not\!#1}
\def\slashb#1{\not\!\!#1}
\def\delsla{\not\!\partial}
\usepackage[dvipdfmx]{xcolor}


\hypersetup{
 setpagesize=false,
 bookmarksnumbered=true,%
 bookmarksopen=true,%
 colorlinks=true,%
 linkcolor=black,
 citecolor=red,
 urlcolor=black,
}
%backreferenceのカスタマイズ. "Back to p.3"のように表示する.
%\renewcommand*{\backref}[1]{(p.#1へ戻る)}
%\newcommand{\backtoc}{\hyperlink{toc}{[目次へ]}}
\newcommand{\backtoc}{\texorpdfstring{\protect\hyperlink{toc}{\hspace{5pt} \scriptsize [目次へ]}}{}}
\newcommand{\mychapter}[1]{\chapter[#1]{#1\backtoc}}
\newcommand{\mysection}[1]{\section[#1]{#1\backtoc}}
\newcommand{\mysubsection}[1]{\subsection[#1]{#1\backtoc}}

\begin{document}
\chapter*{Appendix.A アンジュレータ放射の電場計算}
\cite{wiedemann2015}を参考に導出を行う。

リエナール・ヴィーヘルトポテンシャルは
\begin{eqnarray}
  \bm{E}_r &=& \frac{1}{4\pi \epsilon_0} \frac{e}{c}  \frac{\bm{n}\times \left\{ (\bm{n}-\bm{\beta})\times \bm{\dot{\beta}}\right\}}
  {R(1-\bm{n}\cdot\bm{\beta})^3} \Biggm\vert_r
\end{eqnarray}
これをフーリエ変換することによって、回折格子で波長ごとに分光された電場振幅を求めることができる。
\begin{align}
  \bm{E}(\omega) = \int_{-\infty}^{\infty} \bm{E}_r(t_r) e^{-i\omega t} dt
\end{align}
この時、観測者が観測する振幅を求める必要があるため、積分変数は観測者の時刻である。遅延時間との関係は
\begin{align}
  t = t_r + \frac{R_r}{c}
\end{align}
である。また
\begin{align}
  dt =  (1 - \bm{n}\bm{\beta})dt_r
\end{align}
の関係から、積分変数を変換して、
\begin{align}
  \bm{E}_r(\omega) &= \frac{1}{4\pi\epsilon_0}\frac{e}{c} \int_{-\infty}^{\infty}
   \frac{\bm{n}\times \left\{ (\bm{n}-\bm{\beta})\times \bm{\dot{\beta}}\right\}}{R(1-\bm{n}\cdot\bm{\beta})^2} \Biggm\vert_r
   e^{-i\omega (t_r + R_r /c)} dt_r \\
\end{align}
さらに$R$が一定であると仮定し、
\begin{align}
  \frac{\bm{n}\times \left\{ (\bm{n}-\bm{\beta})\times \bm{\dot{\beta}}\right\}}{R(1-\bm{n}\cdot\bm{\beta})^2} 
   = \frac{d}{dt} \frac{\bm{n}\times \left( \bm{n} \times \bm{\beta}\right) }{1-\bm{n}\cdot\bm{\beta}}
\end{align}
の関係式を用いると
\begin{align}
  \bm{E}(\omega) = \frac{-ie\omega}{4\pi\epsilon_0 cR} 
  \int_{-\infty}^{\infty} \left[ \bm{n}\times \left( \bm{n} \times \bm{\beta}\right) \right]_r e^{-i\omega (t_r + R_r/c)} dt_r
\end{align}



\end{document}