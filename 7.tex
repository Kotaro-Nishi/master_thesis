\documentclass[a4paper,11pt,uplatex]{jsbook}
%\usepackage{fancyhdr}
\setlength{\footskip}{16pt}
\usepackage{amsmath}
\usepackage[dvipdfmx]{graphicx}
\usepackage[dvipdfmx]{color}
%\usepackage{pagecolor}[white]
\usepackage{amsmath,amssymb}
%\usepackage[top=3cm, bottom=3cm, left=3cm, right=3cm]{geometry}
\usepackage{braket}
\usepackage{bm}
\numberwithin{equation}{section}
\usepackage{mathrsfs}
\usepackage{siunitx}
\usepackage{physics}
\usepackage[dvipdfmx]{graphicx}
\usepackage[compat=1.1.0]{tikz-feynhand}
\usepackage{caption}
\usepackage{subcaption}
%\usepackage{cleveref}
\usepackage{float}
\usepackage{multicol}
\setlength{\columnsep}{15mm}
%\usepackage[style=phys,articletitle=false,biblabel=brackets,chaptertitle=false,pageranges=false]{biblatex}
%\usepackage[style=phys]{biblatex}
\usepackage[dvipdfmx]{hyperref}
\usepackage{url}
\usepackage{pxjahyper}
\usepackage{bookmark}
%\usepackage[backref]{hyperref}
\setcounter{tocdepth}{3}
\setlength{\parindent}{2em}
\def\vector#1{\mbox{\boldmath $#1$}}
\def\slash#1{\not\!#1}
\def\slashb#1{\not\!\!#1}
\def\delsla{\not\!\partial}
%\usepackage[dvipdfmx]{xcolor}


\hypersetup{
 setpagesize=false,
 bookmarksnumbered=true,%
 bookmarksopen=true,%
 colorlinks=true,%
 linkcolor=black,
 citecolor=red,
 urlcolor=black,
}
%backreferenceのカスタマイズ. "Back to p.3"のように表示する.
%\renewcommand*{\backref}[1]{(p.#1へ戻る)}
%\newcommand{\backtoc}{\hyperlink{toc}{[目次へ]}}
\newcommand{\backtoc}{\texorpdfstring{\protect\hyperlink{toc}{\hspace{5pt} \scriptsize [目次へ]}}{}}
\newcommand{\mychapter}[1]{\chapter[#1]{#1\backtoc}}
\newcommand{\mysection}[1]{\section[#1]{#1\backtoc}}
\newcommand{\mysubsection}[1]{\subsection[#1]{#1\backtoc}}

\begin{document}
\chapter*{謝辞}
本研究を進めるにあたり、多くの方々にご指導ご助力いただきましたことを心より感謝申し上げます。

指導教官である中村哲教授には、学部から数えて3年の間大変熱心にご指導いただきました。
原子核物理実験の魅力やハイパー核研究の奥深さを教えていただいたことが、この分野に足を踏み入れるきっかけとなりました。
毎週のミーティングやゼミ、研究室での議論を通じて、研究の進め方やハイパー核研究の基礎といった基本的な事柄を学ぶことができました。
特に、研究発表の際のスライド作成や発表の仕方について非常に熱心にご指導いただいたことで、
研究の世界でのコミュニケーション能力を一段と向上させることができたと感じています。
またドイツ、マインツ大学での実験や国際学会を含む学会・研究会の発表といった貴重な研究機会を多くいただきました。
ドイツには累計すると3か月ほど滞在することができ、この間の現地の先生方や学生との共同研究を通じて国際的な環境での研究を経験できたことは、
今後の研究生活で大きな糧になると信じています。本当にありがとうございました。

永尾翔助教には、本研究を進めるうえで多大なるご指導をいただきました。
実験室で実際に手を動かしたり、データ解析をする際に、的確なアドバイスをいただきました。
毎週のミーティングでは、研究の進捗状況や課題について率直な意見をいただき、研究の方向性を見失わないように助けていただきました。
またマインツ大学滞在中には、海外での研究生活や日常生活に不慣れな私に対して、さまざまな面でサポートしていただきました。
ドイツ生活を無事過ごすことができたのは永尾助教のおかげだと思います。本当にありがとうございました。



Pascal Klag氏には、本研究の細部に渡る議論や実験の進め方について多くの助言をいただきました。
マインツ大学において本研究の初期から開発に携わってきたPascal氏の経験や知識は、本研究を進める上でなくてはならないものでした。
ビームタイムでは、Pascal氏が加速器チームなど現地スタッフとのコミュニケーションにも手を貸してくれたことで、円滑に実験を進めることができました。さらに実験の合間にPascal氏とマインツの街を訪れる機会に恵まれたことは非常に貴重な経験でした。
ドイツ生活が楽しいものになり、研究に集中することができたのはこうした要因も大きいと感じています。
また、日本に帰国後にも、定期的にオンラインミーティングやメールでのやり取りを通じて、研究の進捗状況や課題について議論を重ねることができました。英語でのコミュニケーションが必ずしも得意ではない私に対しても、優しくサポートしていただきました。本当にありがとうございました。

Josef Pochodzalla教授には、マインツ大学で実験する機会を与えていただきました。
初めて海外の加速器施設で実験をするということもあり戸惑いも多かったなか、Josef教授が

木野量子氏、Tianhao Shao氏、石毛達大氏、西田賢氏には、特にマインツ大学でのビームタイム中において多大なるご協力をいただきました。
ビームタイムに向けた実験準備や実験中のデータ取得は厳しいスケジュールの中で行われることが多く苦労も絶えませんでしたが、
皆さんと協力して作業を進めることができたことで実験を成功させることができました。
また、研究で行き詰った際には皆さんと議論したり、励まし合ったりすることで、研究のモチベーションを保つことができました。
本当にありがとうございました。
\end{document}