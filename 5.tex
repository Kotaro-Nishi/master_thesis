\documentclass[a4paper,11pt,uplatex]{jsbook}
%\usepackage{fancyhdr}
\setlength{\footskip}{16pt}
\usepackage{amsmath}
\usepackage[dvipdfmx]{graphicx}
\usepackage[dvipdfmx]{color}
%\usepackage{pagecolor}[white]
\usepackage{amsmath,amssymb}
%\usepackage[top=3cm, bottom=3cm, left=3cm, right=3cm]{geometry}
\usepackage{braket}
\usepackage{bm}
\numberwithin{equation}{section}
\usepackage{mathrsfs}
\usepackage{siunitx}
\usepackage{physics}
\usepackage[dvipdfmx]{graphicx}
\usepackage[compat=1.1.0]{tikz-feynhand}
\usepackage{caption}
\usepackage{subcaption}
%\usepackage{cleveref}
\usepackage{float}
\usepackage{multicol}
\setlength{\columnsep}{15mm}
%\usepackage[style=phys,articletitle=false,biblabel=brackets,chaptertitle=false,pageranges=false]{biblatex}
%\usepackage[style=phys]{biblatex}
\usepackage[dvipdfmx]{hyperref}
\usepackage{url}
\usepackage{pxjahyper}
\usepackage{bookmark}
%\usepackage[backref]{hyperref}
\setcounter{tocdepth}{3}
\setlength{\parindent}{2em}
\def\vector#1{\mbox{\boldmath $#1$}}
\def\slash#1{\not\!#1}
\def\slashb#1{\not\!\!#1}
\def\delsla{\not\!\partial}
\usepackage[dvipdfmx]{xcolor}


\hypersetup{
 setpagesize=false,
 bookmarksnumbered=true,%
 bookmarksopen=true,%
 colorlinks=true,%
 linkcolor=black,
 citecolor=red,
 urlcolor=black,
}
%backreferenceのカスタマイズ. "Back to p.3"のように表示する.
%\renewcommand*{\backref}[1]{(p.#1へ戻る)}
%\newcommand{\backtoc}{\hyperlink{toc}{[目次へ]}}
\newcommand{\backtoc}{\texorpdfstring{\protect\hyperlink{toc}{\hspace{5pt} \scriptsize [目次へ]}}{}}
\newcommand{\mychapter}[1]{\chapter[#1]{#1\backtoc}}
\newcommand{\mysection}[1]{\section[#1]{#1\backtoc}}
\newcommand{\mysubsection}[1]{\subsection[#1]{#1\backtoc}}

% 数式
%\usepackage{amsmath,amsfonts}
%\usepackage{bm}
%\usepackage{physics}
% 画像
%\usepackage[dvipdfmx]{graphicx}
%\usepackage[dvipdfmx,colorlinks=true,linkcolor=blue]{hyperref}
%\usepackage{pxjahyper}

\begin{document}

\chapter{結果}
\section{結果}
\subsection{画像処理}
各ピクセルが持つ不定性の結果を示す。
\subsection{波長較正}
水銀灯を用いて波長較正を行った結果を示す。

\subsection{単アンジュレータ}
下流側のアンジュレータのみを用いて取得したデータの解析結果を示す。
このデータを用いてパラメータの較正をおこなった。
\subsubsection{画像の全体的な傾向}
波長依存性があり、これは線形で近似できると考えられる。\\
アンジュレータの位置依存性があり、アンジュレータが下流に移動するにつれて3つの傾向が見られる。
\begin{itemize}
  \item 回折パターンの振幅が上昇する
  \item 回折パターンの形状が変化する
  \item 回折パターンの振幅が周期的に小さく変動する
\end{itemize}
\subsubsection{波長依存性}
波長方向は線形の依存性を仮定した
\subsubsection{位置依存性}
振幅の上昇と周期的な変動がみられる。
\subsubsection{放射光および光学系パラメータ}
回折パターンの形状を決定するパラメータはアンジュレータ - スリット間距離、スリット - カメラ間距離、スリット幅の3つである。
これに加えて放射光関数の情報も形状を決定する。放射光関数の形状は理論式を仮定し、z(U2-slit)、z(slit-cam)、w(slit)の3つのパラメータの最適値を決定する。\\
アンジュレータと光学系の距離に依存して光量が変化する。立体角を考えるとこの光量は距離rに対して$\frac{1}{r^2}$に比例すると考えられる。
また距離によって振幅が周期的に変動する効果は、偏極電磁石による放射だと考えることができる。
\subsubsection{振幅の位置依存性の解析}
\subsubsection{回折パターンの適合度}


\subsection{周期的変化の解析}
2台のアンジュレータの放射光による干渉パターンの解析結果を示す。
\subsubsection{パラメータ決定精度}

\subsection{系統誤差}
\subsubsection{波長依存性}
異なる波長でエネルギーを決定する

\subsubsection{距離依存性}
下流側アンジュレータの位置を4つの区間に分割し、各区間でエネルギーを決定した。
これによりフィット関数が持つ位置依存性による系統誤差を見積もることができる。

\subsubsection{エネルギー依存性}
180.195,210 MeVの3つのエネルギーの結果












\clearpage

\begin{figure}[tb]
  \centering
  \includegraphics[width=0.8\linewidth]{image/1-1.jpg}\\
  \caption{サンプルの図}
  \label{sample_image}
\end{figure}

\begin{itemize}
  \item a
\end{itemize}
\begin{enumerate}
  \item b
\end{enumerate}

\begin{align}
\frac{1}{2} = \qty(\frac{1}{3}) + \qty{1}\Sigma
\end{align}
\end{document}