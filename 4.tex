\documentclass[a4paper,11pt,uplatex]{jsbook}
%\usepackage{fancyhdr}
\setlength{\footskip}{16pt}
\usepackage{amsmath}
\usepackage[dvipdfmx]{graphicx}
\usepackage[dvipdfmx]{color}
%\usepackage{pagecolor}[white]
\usepackage{amsmath,amssymb}
%\usepackage[top=3cm, bottom=3cm, left=3cm, right=3cm]{geometry}
\usepackage{braket}
\usepackage{bm}
\numberwithin{equation}{section}
\usepackage{mathrsfs}
\usepackage{siunitx}
\usepackage{physics}
\usepackage[dvipdfmx]{graphicx}
\usepackage[compat=1.1.0]{tikz-feynhand}
\usepackage{caption}
\usepackage{subcaption}
%\usepackage{cleveref}
\usepackage{float}
\usepackage{multicol}
\setlength{\columnsep}{15mm}
%\usepackage[style=phys,articletitle=false,biblabel=brackets,chaptertitle=false,pageranges=false]{biblatex}
%\usepackage[style=phys]{biblatex}
\usepackage[dvipdfmx]{hyperref}
\usepackage{url}
\usepackage{pxjahyper}
\usepackage{bookmark}
%\usepackage[backref]{hyperref}
\setcounter{tocdepth}{3}
\setlength{\parindent}{2em}
\def\vector#1{\mbox{\boldmath $#1$}}
\def\slash#1{\not\!#1}
\def\slashb#1{\not\!\!#1}
\def\delsla{\not\!\partial}
\usepackage[dvipdfmx]{xcolor}


\hypersetup{
 setpagesize=false,
 bookmarksnumbered=true,%
 bookmarksopen=true,%
 colorlinks=true,%
 linkcolor=black,
 citecolor=red,
 urlcolor=black,
}
%backreferenceのカスタマイズ. "Back to p.3"のように表示する.
%\renewcommand*{\backref}[1]{(p.#1へ戻る)}
%\newcommand{\backtoc}{\hyperlink{toc}{[目次へ]}}
\newcommand{\backtoc}{\texorpdfstring{\protect\hyperlink{toc}{\hspace{5pt} \scriptsize [目次へ]}}{}}
\newcommand{\mychapter}[1]{\chapter[#1]{#1\backtoc}}
\newcommand{\mysection}[1]{\section[#1]{#1\backtoc}}
\newcommand{\mysubsection}[1]{\subsection[#1]{#1\backtoc}}

% 数式
%\usepackage{amsmath,amsfonts}
%\usepackage{bm}
%\usepackage{physics}
% 画像
%\usepackage[dvipdfmx]{graphicx}
%\usepackage[dvipdfmx,colorlinks=true,linkcolor=blue]{hyperref}
%\usepackage{pxjahyper}

\begin{document}

\chapter{データ解析}
\section{モデル関数によるフィッティング}
\subsection{放射光}
アンジュレータ放射光の振幅は放射角の関数として以下のように計算できることが知られている。詳細をAppendixに示す。
また放射光の位相は球面波を仮定する。

\subsection{フレネル回折}
放射光がスリットを通過すると回折を受け、特徴的な縞模様が現れる。回折現象は以下のレイリーゾンマーフェルト積分によって厳密に計算することができる
\begin{eqnarray}
  \text{U}(\text{P}) = \frac{1}{4\pi}\int_{S}\cos(ns)\text{U}(S)\frac{\exp(iks)}{s}\left( ik - \frac{1}{s}\right) -U(S)\frac{\exp(iks)}{s}\left(ik- \frac{1}{r}\right)\cos(nr) dS
\label{レイリーゾンマーフェルト}
\end{eqnarray}
近似① $k \gg 1/r$, $k \gg 1/s$
近似② $\cos(nr) \sim 1$,$\cos(ns) \sim 1$
近似③ 領域Sにおいて$r(S)= z = \text{const.}$,$s(S) = s_0 = \text{const.}$
により式(\ref{レイリーゾンマーフェルト})は
\begin{eqnarray}
  \text{U}(P) = -\frac{i}{2\lambda rs} \int_{S} U(S)\exp ik(r+s) dS
  \label{レイリーゾンマーフェルト近似}
\end{eqnarray}
\begin{eqnarray}
  \text{U}(P) = -\frac{i}{2\lambda rs} \int_{S} U(S)\exp ikr dS
\end{eqnarray}
このような回折現象は伝搬距離によっては近似計算できることが知られている。以下では積分領域$S$上の座標を$x,y$、観測点Pの座標を$x_0,y_0$と表記する。
\begin{eqnarray}
  r &=& \sqrt{z^2 + (x-x_0)^2 + (y-y_0)^2}\\
  &=& z + \frac{1}{2}\frac{(x-x_0)^2 + (y-y_0)^2}{z} - \frac{1}{8}\frac{\left[(x-x_0)^2 + (y-y_0)^2\right]^2}{z^3} +\dots
\end{eqnarray}
$\left[(x-x_0)^2 + (y-y_0)^2\right]^2 \ll z^3$が成立するなら
\begin{eqnarray}
  \text{U}(x_0,y_0) \sim -\frac{i}{2\lambda zs_0}\int \text(U)(x,y) \exp(ik\left\{z +\frac{1}{2}\frac{(x-x_0)^2 + (y-y_0)^2}{z}\right\})
\end{eqnarray}

\subsubsection{数値計算上の計算手法}
数値計算を実行する上では数値積分の手法では、伝搬後のN次元の配列が伝搬前のN次元配列全ての積分を用いて計算されるため計算量は$\text{N}^2$となる。
このような計算コストの高い計算を避けるために、高速フーリエ変換を用いた計算が一般に用いられている。
式(\ref{レイリーゾンマーフェルト近似})を再度$x,y,x_0,y_0$で書き直すと
\begin{eqnarray}
  \text{U}(x_0,y_0) \sim \frac{1}{2i\lambda zs}\int_S \text{U}(x,y) \exp( ik \sqrt{z^2 + (x-x_0)^2 + (y-y_0)^2}) dxdy
\end{eqnarray}
これはカーネル関数$f(x,y) = \sqrt{z^2 +x^2 + y^2}$であるような畳み込みの形で書ける。
\begin{eqnarray}
  U(x_0,y_0) \sim (\text{U} * f)(x,y)
\end{eqnarray}
畳み込みはフーリエ変換を用いることで
\begin{eqnarray}
  (U*f)(x,y) = \mathcal{F}^{-1}(\mathcal{F}(U) \times \mathcal{F}(f))
\end{eqnarray}
と表せる。
\begin{eqnarray}
  \mathcal{F}^{-1}(\mathcal{F}(U) \times \mathcal{F}(f)) &= \mathcal{F}^{-1} \left( \int f(x)e^{iwx}dx \times \int g(x)e^{iwx}dx \right) \\
  &= \mathcal{F}^{-1}\left( \right)
\end{eqnarray}

\subsection{電子ビームサイズ}

\subsection{光学系}
回折格子によって分光された光はレンズによってカメラで収束する

\subsection{パラメータ}
求める関数系は放射光関数と光学系関数からなる。
パラメータの定義を以下に示す。
\begin{itemize}
\item$\gamma$ 電子ビームエネルギーのローレンツ因子
\item$\text{K}$ アンジュレータのK値
\item$\text{z(U2-S)}$下流アンジュレータ-スリット間の距離
\item$\text{z(S-C)}$スリット-カメラ間の距離
\item$\text{w(S)}$スリットの鉛直方向の長さ
\item$\text{y(beam)}$カメラに対するビーム中心のy座標
\item$\text{y(slit)}$カメラに対するスリット中心のy座標
\end{itemize}


\subsection{パラメータ較正}
アンジュレータひとつのデータの解析について

\section{統計誤差の見積もり}
統計誤差の見積もりはブートストラップ法によって行う。
ブートストラップ法はデータから復元抽出を行い、その復元抽出データから統計量を計算することで統計量の分布を推定する手法である。
下流アンジュレータの位置に関してランダムに復元抽出を複数回行い、抽出されたデータの集合に対してフィッティングを行う。得られたパラメータの分布から統計誤差を見積もる。
\section{系統誤差の見積もり}
波長依存性と位置依存性による系統誤差が主要な誤差要因と考えられる。
\subsection{波長依存性}
較正波長(404.65 nm)におけるエネルギー測定の結果に加えて、同じモデル関数を用いて異なる波長でのデータからエネルギーを求める。
異なる波長に対するエネルギーの分布から、モデル関数や装置のもつ波長依存性による系統誤差を見積もる。
\subsection{位置依存性}
アンジュレータの位置によるエネルギーの推定値の違いを見積もる。
具体的には、アンジュレータの位置を4つの区間に分割し、各区間でエネルギーを求める。アンジュレータ位置に対する放射光関数の補正がどの程度の影響を持つかを見積もる。

\section{画像処理}
各runで、アンジュレータが同じ位置にあるときに4枚の画像が取得される。
これらの4枚の画像について、各ピクセルごとに平均値を取り、その値をピクセルの代表値とする。
またピクセルの誤差は4枚のピクセルの標準偏差とする。
周囲のピクセルに対して極端に発光量が多いピクセルはノイズであると判断してマスクし、フィッティングの対象に含まない。
\subsection{平滑化}
画像の平滑化が必要な場合には、波長方向にのみ行う。同様にノイズピクセルをマスクしたうえで平均値と標準偏差を計算し、ピクセルの誤差は標準偏差を計算に用いたピクセル数の平方根で割った値(標準誤差)とする。
\section{下流アンジュレータの解析}
下流アンジュレータのみの磁場によって得られるデータから、放射光関数の位置依存性の情報が抽出できる。
理想的な条件では、アンジュレータが下流に移動すると以下のような変化が見られると考えられる。
\begin{itemize}
  \item 振幅の上昇
  \item 回折パターンの形状の変化
\end{itemize}
振幅の上昇と回折パターンの形状の変化から、距離のパラメータ z(U2-slit),z(slit-cam),w(slits)の最適値を求めることができる。
z(U2-slit),z(slit-cam),w(slit)を固定してフィッティングを行い、適合度がもっとも良いパラメータの組を最適な値として定義する。





\clearpage

\begin{figure}[tb]
  \centering
  \includegraphics[width=0.8\linewidth]{image/1-1.jpg}\\
  \caption{サンプルの図}
  \label{sample_image}
\end{figure}

\begin{itemize}
  \item a
\end{itemize}
\begin{enumerate}
  \item b
\end{enumerate}

\begin{align}
\frac{1}{2} = \qty(\frac{1}{3}) + \qty{1}\Sigma
\end{align}
\end{document}