\documentclass[a4paper,11pt,uplatex]{jsbook}
%\usepackage{fancyhdr}
\setlength{\footskip}{16pt}
\usepackage{amsmath}
\usepackage[dvipdfmx]{graphicx}
\usepackage[dvipdfmx]{color}
%\usepackage{pagecolor}[white]
\usepackage{amsmath,amssymb}
%\usepackage[top=3cm, bottom=3cm, left=3cm, right=3cm]{geometry}
\usepackage{braket}
\usepackage{bm}
\numberwithin{equation}{section}
\usepackage{mathrsfs}
\usepackage{siunitx}
\usepackage{physics}
\usepackage[dvipdfmx]{graphicx}
\usepackage[compat=1.1.0]{tikz-feynhand}
\usepackage{caption}
\usepackage{subcaption}
%\usepackage{cleveref}
\usepackage{float}
\usepackage{multicol}
\setlength{\columnsep}{15mm}
%\usepackage[style=phys,articletitle=false,biblabel=brackets,chaptertitle=false,pageranges=false]{biblatex}
%\usepackage[style=phys]{biblatex}
\usepackage[dvipdfmx]{hyperref}
\usepackage{url}
\usepackage{pxjahyper}
\usepackage{bookmark}
%\usepackage[backref]{hyperref}
\setcounter{tocdepth}{3}
\setlength{\parindent}{2em}
\def\vector#1{\mbox{\boldmath $#1$}}
\def\slash#1{\not\!#1}
\def\slashb#1{\not\!\!#1}
\def\delsla{\not\!\partial}
%\usepackage[dvipdfmx]{xcolor}


\hypersetup{
 setpagesize=false,
 bookmarksnumbered=true,%
 bookmarksopen=true,%
 colorlinks=true,%
 linkcolor=black,
 citecolor=red,
 urlcolor=black,
}
%backreferenceのカスタマイズ. "Back to p.3"のように表示する.
%\renewcommand*{\backref}[1]{(p.#1へ戻る)}
%\newcommand{\backtoc}{\hyperlink{toc}{[目次へ]}}
\newcommand{\backtoc}{\texorpdfstring{\protect\hyperlink{toc}{\hspace{5pt} \scriptsize [目次へ]}}{}}
\newcommand{\mychapter}[1]{\chapter[#1]{#1\backtoc}}
\newcommand{\mysection}[1]{\section[#1]{#1\backtoc}}
\newcommand{\mysubsection}[1]{\subsection[#1]{#1\backtoc}}

% 数式
%\usepackage{amsmath,amsfonts}
%\usepackage{bm}
%\usepackage{physics}
% 画像
%\usepackage[dvipdfmx]{graphicx}
%\usepackage[dvipdfmx,colorlinks=true,linkcolor=blue]{hyperref}
%\usepackage{pxjahyper}

\begin{document}

\chapter{今後の展望}
\section{まとめ}
$E_{beam}=195$~MeVの$\lambda_L=404.65$~nmにおいて
\begin{align}
  E_{beam} = 194.791 \pm 0.019(stat.) ~\text{MeV}
\end{align}
という結果が得られ、$10^{-4}$の水準の統計誤差を達成した。波長依存性による系統誤差の見積もりの結果、
\begin{align}
  E_{beam} = 194.809 \pm 0.039(syst.) ~\text{MeV}
\end{align}
という結果が得られた。

最終的に、$E_{beam}=195$~MeVにおいて、
\begin{align}
  E_{beam} = 194.791 \pm 0.019(stat.) ^{+0.056}_{-0.021}(syst.) ~\text{MeV}
\end{align}
という結果を得た。

$E_{beam} = 180, 210$~MeVにおいても$2\times10^{-4}$以下の誤差でエネルギーを決定した。

波長依存性による系統誤差はモデル関数の改良によって改善されることが期待される。
具体的には、放射光関数の位相・振幅のより厳密な計算や、今回新たに見つかった偏向電磁石による放射光の影響の評価があげられる。
さらに、アンジュレータの位置依存性の結果や、パラメータ同士の強い相関といった問題点を改善することで、現在得られている精度をさらに向上させることができる可能性がある。
単独アンジュレータ、2台のアンジュレータ両方のデータのより詳細な解析が必要である。

波長依存性による系統誤差のほかに、球面波位相の仮定や光子ビームサイズの効果、さらには放射光のコヒーレンスの効果を考えることで、
より厳密な系統誤差の評価ができると期待される。
\section{光学系}
アンジュレータ放射光干渉法では、放射光の波長分光が必要であるため回折格子による分光機能を持った光学系を利用した。
ビームスプリッタを組み合わせて一つの光波から複数の光波を複製することでより複雑な光学解析が可能であると考えらえる。
特にアンジュレータ放射光干渉法では、回折像そのものではなく回折像から復元される元の放射光の情報が求められればよい。
回折格子を通す前の放射光の強度を観測しリファレンスとして利用することで、より精密な光学解析が可能であると考えらえる。

\section{汎用的な電子ビームエネルギー測定手法としての改善点}
今回、200~MeV領域の電子ビームエネルギーを目的として電磁石型のアンジュレータを利用したが、
設計上210~MeVにおいて光学系の較正波長404~nm領域に共鳴波長を持つような放射光を得ることができなかった。
これは電磁石の過電流を避けるためにアンジュレータに流す電流値を抑えたためである。
原理的にはアンジュレータ放射光干渉法は100~MeVから1~GeVまでの広い領域に適用できると考えられるため、
適切な磁場を持つアンジュレータを設計できるとより汎用的な電子ビームエネルギー測定手法として利用できると考えられる。
\subsubsection{永久磁石型アンジュレータ}
永久磁石を用いて、磁石間のギャップを調整することで磁場調整を行うアンジュレータが広く利用されている。
電磁石と比較すると磁場の安定性は高いが、磁場調整の自由度は低くなる。またアンジュレータのギャップの調整には
精密な位置調整が求められるなど、技術的に解決すべき課題が多い。
\subsubsection{アンジュレータ周期の調整}
アンジュレータの偏向定数は式(\ref{eq:K})で与えられるように、
\begin{eqnarray}
  K = 93.4 \frac{B_0}{[\text{T}]}\frac{\lambda_u}{[\text{m}]}
\end{eqnarray}
である。アンジュレータの周期を長くすることで、弱い磁場でも同じ偏向定数を得ることができる。
ただしこれはアンジュレータの巨大化を招き、アンジュレータと光学系の距離依存性が大きくなるという問題点が考えられる。

\section{原子核実験との同時測定}
放射光の測定は、電子ビームを直接測定する必要がないという非破壊性を特徴としている。
この性質を利用して偏向電磁石で発生する放射光を利用したビーム診断技術が素粒子原子核実験で広く使われている。
アンジュレータ放射光干渉法でも、電子ビームエネルギー測定を行ったのちその電子ビームを散乱実験に利用することが原理的に可能である。
これによりビームラインの切り替えといった時間的なロスを避けることができるだけでなく、ビームのエネルギーを連続的に測定することが可能になるという利点もある。

しかしながら、偏向電磁石はビームライン上に固定されているのに対して、
アンジュレータを物理的に移動させると、アンジュレータ内での電子散乱や電子軌道の変化の影響が大きくなることが考えられる。
この影響を見積もるためには、例えばアンジュレータの前後でのビーム位置の測定が考えられる。

また、電子ビームエネルギーの系統誤差は、他の測定手法の結果と比較することによって評価することができる。
この際にも散乱実験など他の手法との同時測定が実現するとより信頼性の高い電子ビームエネルギー測定が可能であると考えられる。
\end{document}